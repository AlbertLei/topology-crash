\begin{center}
  \section{Product topology}
\end{center}

\setcounter{section}{4}

This lecture is about product topology.
There are three important ways of creating new topological spaces from old ones:
forming subspaces, quotient spaces, and product spaces. We focus on product topology today.

\subsection{The product topology: Finite products}


Let $(X, \tau)$ be a topological space.
A collection $\mathcal B$ of open sets of $X$ is said to be a \textit{basis for the topology} $\tau$ if
every open set is a union of members of $\mathcal B$.
So $\mathcal B$ generates the topology $\tau$
in the following sense: if we are told what sets are members of $\mathcal B$, then we can determine the members of
$\tau$ --- they are just all the sets which are unions of members of $\mathcal B$.

Not any collection $\mathcal B$ can form a basis for a topology.

\begin{qst}\label{qst:basis}
  Let $X$ be a non-empty set and let $\mathcal B$ be a collection of subsets of $X$.
  Then $\mathcal B$ is a basis for a topology on $X$ iff (i) $\cup_{B \in \mathcal B} B = X$ and (ii) for all $B_1,B_2 \in \mathcal B$,  $B_1 \cap B_2$ is a member of $\mathcal B$. 
\end{qst}

We use the notion of topology basis to characterize the product topology.


\noindent \textbf{Definition.} 
Let $(X_1, \tau_1)$, $(X_2, \tau_2)$,..., $(X_n, \tau_n)$ be topological spaces.
Then the product topology $\tau$ on the set
$X_1 \times \dots \times X_n$ is the topology having the family $\set{O_1\times \dots \times O_n \colon O_i \in \tau_i}$ as a basis.

\begin{qst}
  Show that the family defined in the above definition constitutes a basis. [Hint: use the result of Question \ref{qst:basis}.]
\end{qst}

An obvious (but incorrect!) candidate for $\tau$
is the set of all sets $O_1 \times ... \times O_n$
where $O_i \in \tau_i$. Unfortunately this is not a topology.

\begin{qst}
  Illustrate that the set of all sets $O_1 \times ... \times O_n$ where $O_i \in \tau_i$ may not be a topology.
\end{qst}
\begin{asw}
  Consider $n=2$ and $X_1 = X_2 = \mathbb R$. The topology defined above does not include the union of
  $(0,1) \times (0,1)$ and $(2,3) \times (2,3)$,
  since this is not $O_1 \times O_2$ for any
  choice of $O_1$ and $O_2$.
\end{asw}

\subsubsection{Projections onto factors of a product}


Recall that given two topologies $\tau_1$ and $\tau_2$ on a set $E$,
$\tau_1$ is said to be \textit{finer} than
$\tau_2$ if $\tau_1 \supseteq  \tau_2$.
For example,
the usual topology on $\mathbb R$ is finer than the
cofinite topology on $\mathbb R$.

Let $(X, \tau_1)$ and $(Y, \tau_2)$ be two topological spaces
and $f$ a mapping from $X$ to $Y$.
Then $f$ is called an open (or rep.\ closed) mapping if for every open (or rep.\ closed) set 
$U$ in $X$, $f(U)$ is open (or rep.\ closed) in $Y$.

\begin{qst}
  Show that $f \colon X \to Y$ being $x \in $ \{open, closed, continuous\} does not imply that it is $y \in$ \{open/closed/closed\} $\setminus \set{x}$.
\end{qst}


Product topology is a nice construction.
Indeed, it is the coarsest topology such that each projection is continuous.


\begin{qst}
Let $(X_1, \tau_1)$, ..., $(X_n, \tau_n)$ be topological spaces and $(X_1 \times \dots \times X_n, \tau)$ their topological space.
For each $i \in \set{1,\dots,n}$, let
$$
p_i \colon X_1 \times \dots \times X_n \to X_i
$$
be the projection mapping; that is,
$p_i( \langle x_1, ..., x_n \rangle  ) = x_i$. Show that
\begin{enumerate}[(a)]
  \item each $p_i$ is a continuous surjective open mapping, and
  \item  $\tau$ is the coarsest topology on the set $X_1 \times \dots \times X_n$ such that each $p_i$ is continuous.
\end{enumerate}
\end{qst}

Given topological spaces $X_1, ..., X_n$, 
the product topology can be defined as the
coarsest topology on $X$ such that each projection is continuous.
This observation is of greater significance in the
discussion of products of an infinite number of topological spaces.

\subsubsection{Tychonoff's theorem}

Tychonoff’s Theorem states that if each $(X_i,\tau_i)$ is compact, then $(X, \tau)$ is compact.
Proving Tychonoff’s Theorem for a general product topology is hard.



\begin{qst}
  Prove the Tychonoff’s Theorem for the product of finite topological spaces. Does the inverse also hold?
\end{qst}

An application of Tychonoff’s Theorem is the 
Heine-Borel Theorem in $\mathbb R^n$. 
A set in $\mathbb R^n$ is compact iff
its projection at each dimension is bounded and closed; that is, the set itself is
bounded and closed.  


\subsubsection{Remarks}

The easiest case of product topology is that of finite products. 
Next we study countably infinite products and then the general case.
The most important result is Tychonoff's theorem.

For those who know some category theory, we observe that the category of
topological spaces and continuous mappings has both products and coproducts.
The products in the category are indeed the products of the topological spaces.
You may care to identify the coproducts.


%% This is a bonus section about proving the fundamental theorem of algebra using topology. 


\subsection{The product topology: countably infinite case}

Intuitively, a curve shall have zero area.
One will be amazing to know the existence of space-filling curves.
We attack the topic of space-filling curves using a
curious space known as Cantor space. 
As an added bonus, you will also have a better understanding of the unit interval $[0,1]$.

From a broader perspective,
in this section we extend our study to countably infinite products
of topological spaces.
In this bigger picture space-filling curves is nothing but one example.

\subsubsection{The product topology}

Let $(X_1, \tau_1), \dots, (X_n, \tau_n), ...$ be a sequence of topological spaces.
The product topology of $\prod X_i$ has the following family as its basis:
$$
\set{ \prod O_i \colon O_i \in \tau_i \text{ for all $i \in \mathbb N$ and $O_i = X_i$ for all but a finite number of }i}.
$$

So a basic open set is of the form: 
$$O_1 \times \dots \times O_n \times X_{n+1} \times X_{n+2} \times \dots .$$

It should be noted that a product of open sets in $X_i$ need not be open in $\prod X_i$. 
For example,  let $X_i = \mathbb R$ and then $(0,1)^\infty :=\prod (0,1)$ is not open in $\mathbb R ^\infty$.

Why, you may (and you should) ask, do we define the product topology in this way?
The answer is that only in this way can we guarantee that the product of compact sets is still compact (Tychonoff's Theorem).

For example, consider the box topology $\tau'$ on $\prod X_i$ that has 
$$\set{\prod_{i=1}^\infty  O_i \colon O_i \in \tau_i \text{ for all }i \in \mathbb N} $$
as its basis.
If each $X_i$ is a finite set with discrete topology, 
then $(\prod X_i, \tau')$ is an infinite discrete space. Moreover, each $X_i$ is compact in $(X_i, \tau_i)$ but
$\prod X_i$ is not compact in $(\prod X_i, \tau')$. 
 

There is another justification, which says the product topology is the coarsest
topology on $\prod X_i$ such that each projection mapping
$p_i \colon X \to X_i$ is continuous. 
The proof is left as an exercise.




\begin{qst}
Let $(X_1, \tau_1), \dots,..., (X_n, \tau_n), \dots$ be a sequence of topological spaces and $(X, \tau)$ be their product space. For each $i$, let $p_i \colon \prod_{j=1}^\infty X_j \to X_i$ be the projection mapping:
$$
p_i (x_1,...x_n,...) = x_i.
$$
Show that
\begin{enumerate}[(a)]
  \item each $p_i$ is a continuous, onto and open mapping;
  \item the product topology $\tau$ is the coarsest topology such that each $p_i$ is continuous.
\end{enumerate}
\end{qst}



\subsubsection{The Cantor space and the Hilbert cube}
The Cantor set $G \subseteq \mathbb R$ is closed, uncountable and of measure zero. 
Arm it with the subspace topology,
we obtain the Cantor space $(G,\tau)$.
We are to show that it is homeomorphic (Tóng Pēi) to a countably infinite product of two-point spaces.


For each $i$, let $(A_i, \tau_i)$ be the set $\set{0,2}$ with the discrete topology. Let $A := \prod A_i$ and
consider the product topology $(A,\tau')$.
Then the map 
\begin{align*}
  f \colon &(G,\tau) \to (A, \tau') \\
           &\sum_{n=1}^\infty \frac{a_n}{3^n} \mapsto (a_1,...a_n,...)
\end{align*}
is a homeomorphism.
Here $\sum_{n=1}^\infty \frac{a_n}{3^n}$ is the ternary representation of a number from the Cantor set.


\begin{qst}
  Show that $f$ is indeed a homeomorphism. That is, it is bijective and continuous.
\end{qst}

\begin{asw}
The properties of Cantor set $G$ imply that $f$ is bijective. As $(G, \tau)$ is compact and $(A, \tau')$ is Hausdorff, $f$ is a homeomorphism if it is continuous.
It's sufficient to show that for any \textit{basic} open set
$U = U_1 \times \dots \times U_N \times A_{N+1} \times A_{N+2} \times \dots$ and any
$a = (a_1,a_2,...) \in U$, 

Consider the open interval
$(\sum_{n=1}^\infty \frac{a_n}{3^n} - \frac{1}{3^{N+2}},  
\sum_{n=1}^\infty \frac{a_n}{3^n} + \frac{1}{3^{N+2}} )$
and let $W$ be the intersection of this open interval with $G$. We have $f(W) \subset U$. \qed
\end{asw}

We showed that the Cantor space is topologically the same with the product space $\set{0,2}^\infty$.
We also suggested that any product of compact spaces is still compact.
From the question above, 
we can show that the product of a countable number of homeomorphic copies of the Cantor Space is homeomorphic to the Cantor Space, and hence is compact.

\begin{prp}
  Let $(G_i, \tau_i)$ be a sequence of topological spaces
  each of which is homeomorphic to the Cantor space $(G,\tau)$. Then
  $$
  (G,\tau) \isom \prod_{i=1}^\infty (G_i, \tau_i) \prod_{i=1}^n (G_i, \tau_i)
  $$
\end{prp}





\newpage

\subsection{The product topology: uncountably infinite case}


Let $I$ be a index set and consider 
$ \set{(E_i, \tau_i) : i \in I}$.
The product space, denoted by $\prod_{i \in I} (X_i, \tau_i)$ consists of the product set and the topology $\tau$ having as its basis the family
$$
\mathcal B = \set{ \prod_{i \in I} O_i: O_i \in \tau_i \text{ and } O_i = X_i \text{ for all but a finite number of } i}.
$$
The topology $\tau$ is called the product topology (or the Tychonoff topology).


\paragraph{Remark.} Although we have defined the product topology
rather differently to the way we did when $I$ was 
infinite or finite,
you should be able to convince
yourself that when $I$ is finite the new definition is equivalent to
our previous ones.


We claim that the Tychonoff theorem still holds but withdraws the proof here.
Interested readers may check a free online book
\textit{Topology Without Tears} by Sidney Morris.

