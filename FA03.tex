\begin{center}
  \section{Compact spaces, Hausdorff spaces, and Locally compact Hausdorff spaces}
\end{center}

\subsection{Compact spaces}

Compactness is an amazing analytical notion expressed in 
topologian.
A topological space 
is compact if all sequences (or more generally, nets) inside it converge as much as possible.
Compactness is a topological notion that was developed to abstract the key property of a subspace of a Euclidean space being closed and bounded: every net must accumulate somewhere in the subspace.
Roughly, the reason is that boundedness implies that
the net cannot escape the subspace, and the point to which it accumulates lies in the subspace by closure. 
This is the statement of the Heine-Borel theorem as you may have seen in calculus.



Compactness provides an \textit{intrinsic} way of formulating this bounded-and-closed property in the context of general topological spaces, without the need to view them as subspaces of an ambient space.
Still, it is common to work with compact subsets of a given space. These are those subsets which are compact spaces with the subspace topology 
(see Question \ref{qst:subspace}).


There are many ways to say that a space $E$ is compact.
The first is perhaps the most common.

\subsubsection{Open covers}

Denote by $E$ a topological space. We say $E$ is \textit{compact} if for any open cover $\set{O_i}_{i \in I}$ with
$\cup_{i \in I} O_i = E$, there exists a finite subcover
$\set{O_j}_{j \in J}$ such that 
$\cup_{j \in J} O_j = E$.
Put in plain English,
a topological space is compact if every open cover has a finite subcover.




An equivalent definition is that for any family of closed sets $\set{F_i}_{i \in I}$ with $\cap_{i \in I} F_i = \emptyset$, there exists $\{F_j\}_{j \in J}$ such that
$\cap_{j \in J} F_j = \emptyset$.
Taking the contrapositive,
if  
\begin{equation}
\cap_{j \in J} F_j \neq \emptyset \text{ for any finite }
\{F_j\}_{j \in J},  
\end{equation}
then the intersection of the set family  
cannot be empty:
$\cap_{i \in I} F_i \neq \emptyset$.

Display (1) is called the \textit{finite intersection property.}
One corollary  of the equivalent definition --- the nested intervals theorem --- is used often in calculus.

\begin{qst}
   Let $E$ be a topological space. Show that if $(F_n)_{n \ge 1}$ is a sequence of nonempty closed sets with $F_n \subseteq F_{n-1}$,
   then $$ \cap_{n \ge 1} F_n \neq \emptyset. $$
\end{qst}  


Usually we wish to determine whether $F \subseteq E$ is 
compact or not, and
we can still use the open covers in $E$ to detect whether 
$F$ is compact or not.

\medskip
\noindent \textbf{Theorem.} Denote by $E$ a topological space and $F\subseteq E$ a subspace.
Then $F$ is compact if and only if
for any open covers of $F$ in $E$, there is a finite subcover.



\begin{qst}\label{qst:subspace}
  Formulate the plain English in the theorem above and
  prove it.
\end{qst}

\subsubsection{Other definitions}


In another direction, the definition of a compact topological space also works for locales.
We can also define
compactness via ultrafilter convergence.
Moreover, A uniform space 
$E$ s compact if and only if it is complete and totally bounded. 
Finally, 
category theorists live in their own world and formulate
compactness using stability properties.
We refer interested readers to nLab:
\url{https://ncatlab.org/nlab/show/compact+space}.

\subsection{Compactum}


Hausdorff spaces (aka ``compacta'') are nice animals.
In a compact Hausdorff space (aka ``compactum''),
every limit of a sequence or more generally of a net that should exist does exist (as it is compact) and does so uniquely (as it is Hausdorff).
Mathematicians love Hausdorff spaces so much that a special name ---compacta/a compactum --- is given to them.\footnote{Indeed, some authors even use ``compact'' to mean ``compact and Hausdorff,''
and use the word ``quasicompact'' to refer to just ``compact'' as we are using it here. 
This custom seems to be prevalent among algebraic geometers, and particularly so within French mathematicians or guys who speak French.}

\begin{qst}\label{qst:compact-closed}
  Denote by $E$ a Hausdorff space and $F \subseteq E$ a  subspce of $E$.
  \begin{enumerate}[(a)]
    \item Show that $F$ is a Hausdorff space.
    \item Show that $F$ is compact $\implies$ $F$ is closed. [Hint one: You need to show $F^c$ is open. Hint two: Show that for each $x$ in $F^c$ there is an open ball in it containing $x$. Note that $F$ is a compactum now!]
    \item Assume furthermore that $E$ is compact. Show that    $F$ is compact $\iff$ $F$ is closed. [Hint: Use the finite intersection property.]
  \end{enumerate}
\end{qst}

%
%\begin{asw}
%  (a) Let $x,y \in F$ and $x \neq y$. 
%      Since $E$ is Hausdorff, there exist two disjoint open sets $U, V \subseteq E$ such that $x \in U$ and $y \in V$.
%      Hence $U \cap F$ and $V \cap F$ are open in $F$ and contain $x$ and $y$ respectively.
%
%  (b)   
%\end{asw}

In $\mathbb R^n$, a set is compact iff it is closed and bounded.
We are interested in the relationship between being compact and closed
(since boundedness is not defined in a topological space).
Question \ref{qst:compact-closed} implies that if a topological space $F$ inhabits a larger Hausdorff space $E$,
then compactness implies closedness.
Moreover, if it inhabits a compactum (i.e.\ being both compact and Hausdorff),
then compactness is the same as closedness.

In general, neither compactness nor closedness implies the other.
See the question below for examples.

\begin{qst}
   Verify the following examples.
   \begin{enumerate}[(a)]
     \item In a topological space with finitely many open sets, each set is compact. For example, the indiscrete/trivial topology.
     \item Endow $R$ with the co-finite topology, in which
        a set is open if and only if its complement is finite. Clearly $N$ is neither open nor closed. 
        Show that $N$ is compact. Indeed, \textit{every} subset is compact in this topology.
     \item Consider \textit{the line with two origins:}
        $E = [-1,0) \cup \set{a,b} \cup (0,1]$. The set 
        $[-1,0) \cup \set{a} \cup (0,1]$ is compact but not closed. Its closure includes point $b$.
   \end{enumerate}
\end{qst}

\subsection{Locally compact Hausdorff space}

Analysts usually work in Hausdorff space, but not so often in compact space (e.g., $\mathbb R$ is non-compact).
Assuming we only talk about stuff in Hausdorff space,
local compactness is
a useful but weaker notion than compactness.
For example, given a compact Hausdorff space $E$,
its subspace $F$ may not be compact but it must be locally compact.


A Hausdorff space $E$ is \textit{locally compact} if every point has a compact neighborhood.
If $E$ is non-Hausdorff, then being locally compact requires each point to have a compact neighborhood basis ---
a neighborhood basis consisting of compact neighborhoods --- which is a stronger condition.
The second definition of local compactness is not used much as we pay little attention to non-Hausdorff space.

\begin{qst}
  Let $E$ be a Hausdorff space. If $E$ is locally compact, then for each $x \in E$ there exists a neighborhood basis consisting of compact neighborhoods for $x$.
\end{qst}

%\begin{asw}
%  Let 
%\end{asw}


In mathematical analysis locally compact spaces that are Hausdorff are of particular interest, which are abbreviated as LCH spaces.
Examples of locally compact Hausdorff spaces that are not compact include:
\begin{itemize}
  \item The Euclidean spaces $\mathbb R^n$, which are are locally compact as a consequence of the Heine--Borel theorem.
  \item All discrete spaces are locally compact and Hausdorff. These are compact only if they are finite.
  \item All open or closed subsets of a locally compact Hausdorff space are locally compact in the subspace topology. This provides several examples of locally compact subsets of Euclidean spaces, such as the open unit ball.

 
\end{itemize}
